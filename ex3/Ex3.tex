\documentclass{article}
\usepackage{graphicx}
\usepackage{amsmath}
\usepackage{listings}
\usepackage{caption}
\usepackage[hidelinks]{hyperref}
\setlength{\parindent}{0pt}


\begin{document}
    

\begin{titlepage}
    \vbox{ }
    \vbox{ }
    \begin{center}
        % Set course here
        \includegraphics[width=0.40\textwidth]{img/NTNU_logo.png}\\[1cm]
    \textsc{\Large TMA4135 - Matematikk 4D}\\[0.5cm]
    \vbox{ }
    
    % Set title here
    { \huge \bfseries Execise \#1}\\[0.4cm]
    
    \large
    \emph{Author:}\\
    Sondre Pedersen
    \vfill
    
    {\large\today}
\end{center}
\end{titlepage}

    \captionsetup[figure]{labelformat=empty}

    
    \section*{\textbf{Oppgave 1}}
    \vspace*{12pt}\small\textbf{a)}
    \begin{figure*}[ht]
        \centering
        \includegraphics*[width=\textwidth]{img/1a.PNG}
        \caption{Mulighetsområdet med en vilkårlig valgt målfunksjon}
    \end{figure*}
    
    \vspace*{12pt}\small\textbf{b)}
    
    
    Introduserer $x_1' = x_1 + 2$, slik at restriksjon (5) ikke trenger en slakk-variabel. Skriver den nye variabelen uten merking nedenfor.
    \[
    \begin{array}{rrrrrrrl}
        z & -3(x_1 + 2) & -x_2 &&&&& = 0 \\ 
        & +(x_1 + 2) & -x_2 & +s_1 &&&& = 5 \\ 
        & +3(x_1 + 2) & -2x_2 &&+s_2&&& = 18 \\ 
        & +4(x_1 + 2) & +2x_2 &&&-s_3&& = 9 \\ 
        && +x_2 &&&&+s_4& = 6 \\ 
    \end{array}
    \]
    Forenkles til
    \[
    \begin{array}{rrrrrrrl}
        z & -3x_1 & -x_2 &&&&& = 6 \\ 
        & +x_1 & -x_2 & +s_1 &&&& = 3 \\ 
        & +3x_1 & -2x_2 &&+s_2&&& = 12 \\ 
        & +4x_1 & +2x_2 &&&-s_3&& = 1 \\ 
        && +x_2 &&&&+s_4& = 6 \\ 
    \end{array}
    \]
                
    \vspace*{12pt}\small\textbf{c)}
                

\end{document}