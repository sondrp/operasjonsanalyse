\documentclass{article}
\usepackage{graphicx}
\usepackage{amsmath}
\usepackage{listings}
\usepackage{caption}
\usepackage[table]{xcolor}
\usepackage[hidelinks]{hyperref}
\setlength{\parindent}{0pt}


\begin{document}
    

\begin{titlepage}
    \vbox{ }
    \vbox{ }
    \begin{center}
        % Set course here
        \includegraphics[width=0.40\textwidth]{img/NTNU_logo.png}\\[1cm]
    \textsc{\Large TMA4135 - Matematikk 4D}\\[0.5cm]
    \vbox{ }
    
    % Set title here
    { \huge \bfseries Execise \#1}\\[0.4cm]
    
    \large
    \emph{Author:}\\
    Sondre Pedersen
    \vfill
    
    {\large\today}
\end{center}
\end{titlepage}

    \section*{\textbf{Oppgave 1}}
    \small\textbf{a)}
    \begin{figure*}[ht]
        \includegraphics*[width=\linewidth]{img/1a.PNG}
    \end{figure*}
    
    
    Her er alle hjørnepunktene som omfavner det beige området en lovlig løsning. (0, 0), (0, 1500), (250, 1500), (650, 1100), (1000, 400), (1000, 0).

    \small\textbf{b)}
    
    \begin{table}[h!]
        \centering
        \begin{tabular}{|c|c|c|}
            \hline
            $x_1$ & $x_2 $ & $12x_1 + 9x_2$ \\\hline
            0 & 0 & 0 \\\hline
            0 & 1500 & 13500 \\\hline
            250 & 1500 & 16500 \\\hline
            650 & 1100 & 17700 \\\hline
            1000 & 400 & 15600 \\\hline
            1000 & 0 & 12000 \\\hline
        \end{tabular}
    \end{table}
    
    Vi vet at dersom en optimal løsning finnes, er denne løsningen et av hjørnepunktene. Siden $x_1 = 650$ $x_2 = 1100$ er den beste løsningen blant alle lovlige hjørnepunkt,
    følger det at det er den beste løsningen på problemet. Målfunksjonsverdien blir da 17700. De bindende restriksjonene er $4x_1 + 2x_2 <= 4800$ (blå) og $x_1 + x_2 <= 1750$ (rød).
    
    \begin{figure*}
        \centering
        \includegraphics*[width=\linewidth]{img/1b.PNG}
        \caption*{Målfunksjon gjennom optimal verdi}
    \end{figure*}
    
    \small\textbf{c)}
    
    Simplex-metoden vil følge denne stien: (0, 0) -- (1000, 0) -- (1000, 400) -- (650, 1100). Den baserer seg på hvilken retning som gir raskest økning i z per enhet.
    Den vil stoppe i (650, 1100), ettersom ingen nabo øker målfunksjonsverdien.
    
    \small\textbf{d)}
    
    \begin{align*}
        x_1 + s_1 &= 1000 \\
        x_2 + s_2 &= 1500 \\
        x_1 + x_2 + s_3 &= 1750 \\
        4x_1 + 2x_2 + s_4 &= 4800 \\
        x_1, x_2, s_1, s_2, s_3, s_4 &>= 0
    \end{align*}

    
    \begin{figure*}[ht]
        \centering
        \includegraphics*[width=\linewidth]{img/1d.PNG}
        \caption*{Graf i a) der restriksjoner uttrykkes ved slakkvariablene.}
    \end{figure*}

    \small\textbf{e)}
    
    \begin{table}[h!]
        \centering
        \begin{tabular}{|c|c|c|c|c|c|}
            \hline
            $x_1$ & $x_2 $ & $s_1$ & $s_2$ & $s_3$ & $s_4$ \\\hline
            0 & 0 & 1000 & 1500 & 1750 & 4800 \\\hline
            0 & 1500 & 1000 & 0 & 250 & 1800 \\\hline
            250 & 1500 & 750 & 0 & 0 & 800 \\\hline
            650 & 1100 & 350 & 400 & 0 & 0 \\\hline
            1000 & 400 & 0 & 1100 & 350 & 0 \\\hline
            1000 & 0 & 0 & 1500 & 750 & 800 \\\hline
        \end{tabular}
    \end{table}
    
    Tabellen viser basisløsningen for alle lovlige verdier. 
    
    Restriksjon (2) og (4) møtes i punkt (450, 1500). Her blir basisløsningen 450, 1500, 550, 0, -200, 0.
    
    Vi kan se at denne løsningen ikke er gyldig siden en av slakkvariablene er negativ. Dette er det samme som å bruke for mye av en ressurs. 

    \small\textbf{f)}

    $s_3$ og $s_4$ er ved nedre grense i optimal løsning. $x_1, x_2, s_1, s_2$ er da basis.

    
    \section*{\textbf{Oppgave 2}}
    \small\textbf{a)}
    \begin{figure*}[ht]
        \includegraphics*[width=\linewidth]{img/2a.PNG}
    \end{figure*}
    
    
    Alle hjørnepunktløsninger er markert. Tillatte er grønne, mens ikke-tillatte er røde.

    \small\textbf{b)}
    
    \begin{table}[h!]
        \centering
        \begin{tabular}{|c|c|c|}
            \hline
            $x_1$ & $x_2 $ & z = $x_1 + 2x_2$ \\\hline
            0 & 0 & 0 \\\hline
            0 & 8/3 & 16/3 \\\hline
            2 & 2 & 6 \\\hline
            4 & 0 & 4 \\\hline
        \end{tabular}
    \end{table}
    
    Vi ser at $x_1 = 2, x_2 = 2$ gir optimal løsning. 
    
    \small\textbf{c)}
    
    Sekvensen som brukes av Simplex metoden er (0, 0) -- (0, 8/3) -- (2, 2). Her antar vi at iterasjonen starter i (0, 0).
    
    \pagebreak\small\textbf{d)}

    \begin{align}
        z - x_1 - 2x_2 &= 0 \\ 
        x_1 + 3x_2 + s_1 &= 8 \\
        x_1 + x_2 + s_2 &= 4
    \end{align}

    \small\textbf{e)}
    
    \begin{table}[h!]
        \centering
        \begin{tabular}{|c|c|c|c|c|c|c|c|}
            \hline
            Løsning index & $x_1$ & $x_2 $ & $s_1$ & $s_2$ & Basis & Ikke-basis & z \\\hline
            1 & 0 & 0 & 8 & 4 & $s_1, s_2$ & $x_1, x_2$ & 0 \\\hline
            2 & 0 & 8/3 & 0 & 4/3 & $x_2, s_2$ & $x_1, s_1$ & 16/3 \\\hline
            3 & 2 & 2 & 0 & 0 & $x_1, x_2$ & $s_1, s_2$ & 6 \\\hline
            4 & 4 & 0 & 4 & 0 & $x_1, s_1$ & $x_2, s_2$ & 4 \\\hline
        \end{tabular}
    \end{table}

    \small\textbf{f)}
    
    Demonstrerer at basisløsningen oppfyller likningssystemet ved å sette inn verdiene i likning (1), (2), (3)
    
    \vspace*{12pt}\textbf{Løsning 1}
    
    (1): 0 - 0 - 2 $\times$ 0 = 0
    
    (2): 0 + 3 $\times$ 0 + 8 = 8
    
    (3): 0 + 0 + 4 = 4
    
    \vspace*{12pt}\textbf{Løsning 2}
    
    (1): 16/3 - 0 - 2 $\times$ 8/3 = 0

    (2): 0 + 3 $\times$ 8/3 + 0 = 8
    
    (3): 0 + 8/3 + 4/3 = 4
    
    \vspace*{12pt}\textbf{Løsning 3}
    
    (1): 6 - 2 - 2 $\times$ 2 = 0
    
    (2): 2 + 3 $\times$ 2 + 0 = 8
    
    (3): 2 + 2 + 0 = 4
    
    \vspace*{12pt}\textbf{Løsning 4}
    
    (1): 4 - 4 - 2 $\times$ 0 = 0
    
    (2): 4 + 3 $\times$ 0 + 4 = 8
    
    (3): 4 + 0 + 0 = 4
    
    \vspace*{12pt}
    Siden alle uttrykkene er gyldige, vet vi at basis-løsningen er en løsning på likningssystemet.
    
    \vspace*{12pt}\small\textbf{g)}
    
    
    \begin{table}[h!]
        \centering
        \begin{tabular}{|c|c|c|c|c|c|c|c|}
            \hline
            Løsning index & $x_1$ & $x_2 $ & $s_1$ & $s_2$ & Basis & Ikke-basis & z \\\hline
            5 & 0 & 4 & -4 & 0 & $s_1, x_2$ & $x_1, s_2$ & 8 \\\hline
            6 & 8 & 0 & 0 & -4 & $x_1, s_2$ & $x_2, s_1$ & 8 \\\hline
        \end{tabular}
    \end{table}
    
    \vspace*{12pt}\small\textbf{h)}
    
    \vspace*{12pt}\textbf{Løsning 5}
    
    (1): 8 - 0 - 2 $\times$ 4 = 0
    
    (2): 0 + 3 $\times$ 4 + -4 = 8
    
    (3): 0 + 4 + 0 = 4
    
    \vspace*{12pt}\textbf{Løsning 6}
    
    (1): 8 - 8 - 2 $\times$ 0 = 0
    
    (2): 8 + 3 $\times$ 0 + 0 = 8
    
    (3): 8 + 0 + -4 = 4
    
    \vspace*{12pt}\small\textbf{i)}
    
    For å systematisere den algebraiske metoden har jeg gjort noen endringer fra det som ble gjennomgått i video. Jeg fører opp hele problemet (på utvidet form) i en matrise, men utfører bare Gauss-Jordan på kolonnene som er i basis (hvite).
    Kolonnene som ikke er i basis (grå) "henger bare med", slik at de kan brukes i neste iterasjon. Kolonnene representerer koeffisientene til $z, x_1, x_2, s_1, s_2$ og konstanten i likning (1), (2), (3). 
    
    \vspace*{12pt}Starter med $x_1, x_2$ som ikke-basis. 
    
    \begin{align*}
        \left[
            \begin{array}{c>{\columncolor{gray!30}}c>{\columncolor{gray!30}}ccc|c}
            1 & -1 & -2 & 0 & 0 & 0 \\
            0 & 1  & 3 & 1 & 0 & 8 \\
            0 & 1  & 1 & 0 & 1 & 4
        \end{array}
        \right]
    \end{align*}
    
    Løsningen er ikke optimal, fordi en økning i disse vil øke z. 
    Tar $x_2$ inn i basis fordi koeffisienten er større for $x_2$ enn for $x_1$. Tar $s_1$ ut av basis fordi $\frac{8}{3} < \frac{4}{1}$
    
    \begin{align*}
        \left[
            \begin{array}{c>{\columncolor{gray!30}}cc>{\columncolor{gray!30}}cc|c}
                1 & -1 & -2 & 0 & 0 & 0 \\
                0 & 1  & 3 & 1 & 0 & 8 \\
                0 & 1  & 1 & 0 & 1 & 4
            \end{array}
            \right]
            & \Rightarrow
            \left[
                \begin{array}{c>{\columncolor{gray!30}}cc>{\columncolor{gray!30}}cc|c}
                    1 & -1 & -2 & 0 & 0 & 0 \\
                    0 & 1/3 & 1 & 1/3 & 0 & 8/3 \\
                    0 & 1 & 1 & 0 & 1 & 4
                \end{array}
                \right]
                & \Rightarrow
                \left[
                    \begin{array}{c>{\columncolor{gray!30}}cc>{\columncolor{gray!30}}cc|c}
                        1 & -1/3 & 0 & 2/3 & 0 & 16/3 \\
                        0 & 1/3 & 1 & 1/3 & 0 & 8/3 \\
        0 & 2/3 & 0 & -1/3 & 1 & 4/3
        \end{array}
        \right]
    \end{align*}
    
    Forklaring: 
    
    Steg 1: (ii) = (ii) / 3
    
    Steg 2: (i) = (i) + 2(ii), (iii) = (iii) - (ii)
    
    \vspace*{12pt}Løsningen er ikke optimal, fordi $A_{12}$ er negativ. Tar dermed $x_1$ inn i basis.
    Tar $s_2$ ut av basis, ettersom $\frac{4/3}{2/3} < \frac{8/3}{1/3}$.
    
    \begin{align*}
        \left[
            \begin{array}{ccc>{\columncolor{gray!30}}c>{\columncolor{gray!30}}c|c}
                1 & -1/3 & 0 & 2/3 & 0 & 16/3 \\
                0 & 1/3 & 1 & 1/3 & 0 & 8/3 \\
                0 & 2/3 & 0 & -1/3 & 1 & 4/3
            \end{array}
            \right]
            & \Rightarrow
            \left[
                \begin{array}{ccc>{\columncolor{gray!30}}c>{\columncolor{gray!30}}c|c}
                    1 & -1/3 & 0 & 2/3 & 0 & 16/3 \\
                    0 & 1 & 0 & -1/2 & 3/2 & 2 \\
                    0 & 1/3 & 1 & 1/3 & 0 & 8/3
                \end{array}
                \right]
                & \Rightarrow
                \left[
                    \begin{array}{ccc>{\columncolor{gray!30}}c>{\columncolor{gray!30}}c|c}
                        1 & 0 & 0 & 5/6 & 1/2 & 6 \\
                        0 & 1 & 0 & -1/2 & 3/2 & 2 \\
                        0 & 0 & 1 & 1/2 & -1/2 & 2
                    \end{array}
                    \right]
                \end{align*}
                
    Forklaring: 
    
    Steg 1: (iii) = 3(iii) / 2, bytter plass (ii) og (iii) 

    Steg 2: (i) = (i) + (ii) / 2, (iii) = (iii) - (ii) / 2
    \vspace*{12pt}
    
    Denne løsningen er optimal. Både $A_{14}$ og $A_{15}$ er positive. Kan lese av resultatet: z = 6, $(x_1, x_2, s_1, s_2) = (2, 2, 0, 0)$.
    
    
    \vspace*{12pt}\small\textbf{j)}

    Jeg oppdager nå at simplex-metoden i tabellform er \textit{veldig} lik min variant av den algebraiske formen. Jeg ser derfor ikke hensikten med å gjøre oppgaven på nytt, spesielt ved tanke på at det tar
    fryktelig lang tid å føre inn i Latex. Jeg får heller bruke tabellformen en annen gang.



\end{document}